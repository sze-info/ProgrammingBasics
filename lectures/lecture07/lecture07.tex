\documentclass[usenames,dvipsnames,aspectratio=169]{beamer}
\usepackage{../common/prgBasics}

\title[Lecture 7.]{Programming basics}
\subtitle{(GKNB\_INTA023)}

\begin{document}

%1
\begin{frame}[plain]
  \titlepage
\end{frame}

%2
\begin{frame}{Searching for the longest 3D vectors}
  \begin{exampleblock}{\textattachfile{vector1.c}{vector1.c}}
    \scriptsize
    \lstinputlisting[style=c,linerange={1-18}]{vector1.c}
  \end{exampleblock}
\end{frame}

%3
\begin{frame}{Searching for the longest 3D vectors}
  \begin{exampleblock}{\textattachfile{vector1.c}{vector1.c}}
    \scriptsize
    \lstinputlisting[style=c,firstline=19,firstnumber=19]{vector1.c}
  \end{exampleblock}
  \vfill
  Problem:
  \begin{itemize}
    \item the X, Y, Z coordinates of a vector are more closely related than eg. the X coordinates of various vectors
    \item but our arrays do not reflect it
  \end{itemize}
\end{frame}

%4
\begin{frame}{Structures}
  Main features:
  \begin{itemize}
    \item Easy handling of a group of logically related variables
    \item A compound, user-defined type can be created
    \item A group of one or more \emph{members} with unique identifiers
    \item Possibilities:
    \begin{itemize}
      \item Assignment (copy)
      \item Can be passed to functions
      \item Can be the return value of a function
    \end{itemize}
    \item Impossible: comparison (possibly per member)
    \item \emph{Almost} anything can become a member
  \end{itemize}
\end{frame}

%5
\begin{frame}{Searching for the longest 3D vectors}
  \begin{exampleblock}{\textattachfile{vector2.c}{vector2.c}}
    \scriptsize
    \lstinputlisting[style=c,linerange={1-16},]{vector2.c}
  \end{exampleblock}
\end{frame}

%6
\begin{frame}{Searching for the longest 3D vectors}
  \begin{exampleblock}{\textattachfile{vector2.c}{vector2.c}}
    \scriptsize
    \lstinputlisting[style=c,linerange={17-32},firstnumber=17]{vector2.c}
  \end{exampleblock}
\end{frame}

%7
\begin{frame}[fragile]{Structure declaration}
  \footnotesize
  General usage: \textbf{struct} \emph{{<}structure-tag>}\\
      \qquad \emph{{<}{member-declarations}> <variable-declarations>};
  \begin{exampleblock}{Example structure declaration}
    \begin{verbatim}
struct student { // Structure declaration
  char name[64];
  int pointsEarned;
};

struct student Jane, as[1000]; // Variable declarations
\end{verbatim}
  \end{exampleblock}
  \begin{itemize}
    \item \texttt{student} is the tag of the structure, it identifies the type together with keyword \texttt{struct}:\\
      \texttt{struct student Jane;}
    \item Members: \texttt{name, pointsEarned} (unique identiers (names) inside the structure)
    \item Variables: \texttt{Jane}\\ 
      \texttt{struct student as[1000];} an array of 1000 students
  \end{itemize}
\end{frame}

%8
\begin{frame}[fragile]{Structure declaration}
  Where should a structure be \emph{declared}?
  \begin{itemize}
    \item In front of the first usage of the type
    \item Generally at the beginning of the source code, outside of all functions
  \end{itemize}
  All declarations create a \emph{new and unique type} even if their members are the same
  \scriptsize
  \begin{exampleblock}{Different types}
    \begin{verbatim}
struct student1 {
  char name[64]; int pointsEarned;
};

struct student2 {
  char name[64]; int pointsEarned;
};

struct student1 Jane;
struct student2 Joe;
Jane = Joe; // error: incompatible types when assigning to type 
            // 'struct student1' from type 'struct student2'
\end{verbatim}
  \end{exampleblock}
  \normalsize
  Where should a structure be \emph{defined}? $\to$ In the narrowest possible scope
\end{frame}

%9
\begin{frame}[fragile]{Structure member declaration}
  \small
  \begin{itemize}
    \item A member can be eg.
    \begin{itemize}
      \footnotesize
      \item an already declared structure
      \item an embedded structure, even without tag
      \item array
      \item (a function pointer)
    \end{itemize}
    \item The name of the member must be unique only inside the structure
    \item The semicolon (;) at the end of the declaration cannot be omitted!
  \end{itemize}
  \small
  \begin{exampleblock}{Valid member declarations}
    \begin{verbatim}
struct s { int i; };
struct member_decl {
  struct s s1;
  struct { int i; long l; } e;
  int numbers[30];
};
\end{verbatim}
  \end{exampleblock}
\end{frame}

%10
\begin{frame}[fragile]{Structure member declaration}
  A member's type cannot be eg.
  \begin{itemize}
    \item \texttt{void}
    \item itself
    \item function
  \end{itemize}
  \scriptsize
  \begin{alertblock}{Invalid member declarations}
    \begin{verbatim}
struct incomplete;
struct member_error {
  void v; /* error: variable or field ‘v’ declared void */
  struct incomplete s; /* error: field ‘s’ has incomplete type */
  struct member_error me; /* error: field ‘me’ has incomplete type */
};
\end{verbatim}
  \end{alertblock}
  \tiny
  Remark: an incomplete array ($\to$ its size is unknown to the compiler) can be a member according to the C99 standard, if certain conditions are met.
\end{frame}

%11
\begin{frame}[fragile]{Accessing structure members}
  Member access operator
  \begin{itemize}
    \item \emph{structure\kiemel{.}member}
    \item High precedence operator, the direction of associativity is from left to right
  \end{itemize}
  \footnotesize
  \begin{exampleblock}{Accessing structure members, assignments}
    \begin{verbatim}
struct student {
  char name[64];
  char neptun[7];
  struct {
    int day, month, year;
  } birth;
};
/* ... */
struct hallgato Jane;
strcpy(Jane.name, "Jane Doe");
strcpy(Jane.neptun, "A1B2C3");
Jane.birth.day = 2; Jane.birth.month = 1; 
Jane.birth.year = 1990;
\end{verbatim}
  \end{exampleblock}
\end{frame}

%12
\begin{frame}[fragile]{Initialization of structures}
  The members are initialized one after another to the values in the initializer list.
  A structure of the same type can also be az initializer.
  \begin{exampleblock}{Initialization of structures}
    \begin{verbatim}
struct student {
  char name[64], neptun[7];
  int day, month, year;
};

struct student Jane = 
  { "Jane Doe", "A1B2C3", 23, 4, 1990 };
struct student Mary = Jane;
\end{verbatim}
  \end{exampleblock}
\end{frame}

%13
\begin{frame}[fragile]{Initialization of structures}
  \small
  Initialization of embedded structures: with embedded initializers
  \begin{exampleblock}{Initialization of an embedded structure and array}
    \scriptsize
    \begin{verbatim}
struct date {
  int day, month, year;
};

struct student {
  char name[64], neptun[7];
  struct date birth, graduation;
};

struct student Jane = { "Jane Doe", "A1B2C3", 
  {23, 4, 1990}, {3, 6, 2015} };
\end{verbatim}
  \end{exampleblock}
  \begin{itemize}
    \item The count of initializer list elements must not exceed the number of structure members!
    \item If it has fewer elements $\to$ all remaining bits are going to be set to zero
    \item In case of embedded types the \texttt{\{ \}} can be omitted or can be even placed around all initializers, but it is recommended to follow the internal structure of the type.
  \end{itemize}
\end{frame}

%14
\begin{frame}[fragile]{Initialization of structures}
  Usage of \emph{designators}: direct references to the members (C99)
  \begin{exampleblock}{Initialization of an embedded structure and arrays with designators}
    \footnotesize
    \begin{verbatim}
struct student {
  char name[64], neptun[7];
  struct {
    int day, month, year;
  } birth;
} Jane = { .name="Jane Doe", .neptun="A1B2C3",
           {23, 4, 1990} };
\end{verbatim}
  \end{exampleblock}
  In case of a missing designator the initialization continues with the member that follows the member referenced by a designator for the last time. The order of designator usage is arbitrary.
\end{frame}

%15
\begin{frame}{Date management}
  \begin{exampleblock}{\textattachfile{calendar1.c}{calendar1.c}}
    \scriptsize
    \lstinputlisting[style=c,linerange={5-21},firstnumber=5]{calendar1.c}
  \end{exampleblock}
\end{frame}

%16
\begin{frame}{Date management}
  \begin{exampleblock}{\textattachfile{calendar1.c}{calendar1.c}}
    \scriptsize
    \lstinputlisting[style=c,linerange={23-36},firstnumber=23]{calendar1.c}
  \end{exampleblock}
\end{frame}

%17
\begin{frame}{Date management}
  \begin{exampleblock}{\textattachfile{calendar1.c}{calendar1.c}}
    \scriptsize
    \lstinputlisting[style=c,linerange={38-52},firstnumber=38]{calendar1.c}
  \end{exampleblock}
\end{frame}

%18
\begin{frame}{Date management}
  \begin{exampleblock}{\textattachfile{calendar1.c}{calendar1.c}}
    %\footnotesize
    \small
    \lstinputlisting[style=c,linerange={54-63},firstnumber=54]{calendar1.c}
  \end{exampleblock}
\end{frame}

%19
\begin{frame}{Date management}
  \begin{exampleblock}{\textattachfile{calendar1.c}{calendar1.c}}
    \footnotesize
    \lstinputlisting[style=c,linerange={65-79},firstnumber=65]{calendar1.c}
  \end{exampleblock}
\end{frame}

%20
\begin{frame}[fragile]{Date management}
  \begin{block}{Output}
    \begin{verbatim}
The given date is valid.
23.10.2020 is the 297th day of the year.
How many days are left to christmas? 62
The 300th day of 2020 is: 26.10
\end{verbatim}
  \end{block}
\end{frame}

%21
\begin{frame}[fragile]{Drawing rectangles}
  \tiny
  \begin{block}{Output (1/2)}
    \begin{verbatim}
Please enter the data of rectangles!
X coordinate of the top left corner of rectangle #1: [0, 78] (enter a negative value to exit) 1
Y coordinate of the top left corner rectangle #1 [0, 23] 1
X coordinate of the bottom right corner rectangle #1 [2, 79] 11
Y coordinate of the bottom right corner rectangle #1 [2, 24] 11
Drawing character of rectangle #1: |
X coordinate of the top left corner of rectangle #2: [0, 78] (enter a negative value to exit) 6
Y coordinate of the top left corner rectangle #2 [0, 23] 6
X coordinate of the bottom right corner rectangle #2 [7, 79] 16
Y coordinate of the bottom right corner rectangle #2 [7, 24] 16
Drawing character of rectangle #2: +
X coordinate of the top left corner of rectangle #3: [0, 78] (enter a negative value to exit) 15
Y coordinate of the top left corner rectangle #3 [0, 23] 2
X coordinate of the bottom right corner rectangle #3 [16, 79] 30
Y coordinate of the bottom right corner rectangle #3 [3, 24] 7
Drawing character of rectangle #3: -
X coordinate of the top left corner of rectangle #4: [0, 78] (enter a negative value to exit) -1
...
\end{verbatim}
  \end{block}
\end{frame}

%22
\begin{frame}[fragile]{Drawing rectangles}
  \footnotesize
  \begin{block}{Output (2/2)}
    \begin{verbatim}
...
 |||||||||||                                                                   
 |||||||||||   ----------------                                                
 |||||||||||   ----------------                                                
 |||||||||||   ----------------                                                
 |||||||||||   ----------------                                                
 |||||+++++++++----------------                                                
 |||||+++++++++----------------                                                
 |||||+++++++++++                                                              
 |||||+++++++++++                                                              
 |||||+++++++++++                                                              
 |||||+++++++++++                                                              
      +++++++++++                                                              
      +++++++++++                                                              
      +++++++++++                                                              
      +++++++++++                                                              
      +++++++++++
...
\end{verbatim}
  \end{block}
\end{frame}

%23
\begin{frame}{Drawing rectangles}
  \begin{exampleblock}{\textattachfile{rectangle1.c}{rectangle1.c}}
    \footnotesize
    \vspace{-.3cm}
    \lstinputlisting[style=c,linerange={1-17},firstnumber=1]{rectangle1.c}
    \vspace{-.3cm}
  \end{exampleblock}
\end{frame}

%24
\begin{frame}{Drawing rectangles}
  \begin{exampleblock}{\textattachfile{rectangle1.c}{rectangle1.c}}
    \tiny
    \vspace{-.3cm}
    \lstinputlisting[style=c,linerange={48-73},firstnumber=48]{rectangle1.c}
    \vspace{-.3cm}
  \end{exampleblock}
\end{frame}

%25
\begin{frame}{Drawing rectangles}
  \begin{exampleblock}{\textattachfile{rectangle1.c}{rectangle1.c}}
    \small
    \lstinputlisting[style=c,linerange={38-46},firstnumber=38]{rectangle1.c}
  \end{exampleblock}
\end{frame}

%26
\begin{frame}{Drawing rectangles}
  \begin{exampleblock}{\textattachfile{rectangle1.c}{rectangle1.c}}
    \scriptsize
    \vspace{-.3cm}
    \lstinputlisting[style=c,linerange={18-36},firstnumber=18]{rectangle1.c}
    \vspace{-.3cm}
  \end{exampleblock}
\end{frame}

\end{document}
